% DOCUMENT SETUP
\onehalfspacing
\widowpenalty10000
\clubpenalty10000

% INHALTSVERZEICHNIS SETUP
\cftsetindents	{section}{0em}{4em}
\cftsetindents	{subsection}{0em}{4em}
\cftsetindents	{subsubsection}{0em}{4em}
\renewcommand	{\contentsname}{Inhaltsverzeichnis}
\setcounter		{tocdepth}{3}
\setcounter		{secnumdepth}{5}

% ABBILDUNGEN UND TABELLEN SETUP

% Abbildungen usw. werden nach kapitel.lfd nummeriert
\counterwithin{figure}{section}
\counterwithin{table}{section}

%\renewcommand	{\listfigurename}{Abbildungsverzeichnis}
%\renewcommand	{\listtablename}{Tabellenverzeichnis}

%\addto{\captionsngerman}{
%    \renewcommand*{\figurename}{Abb.}
%    \renewcommand*{\tablename}{Tab.}
%}

%\addtocontents{lof}{\linespread{2}\selectfont}
%\addtocontents{lot}{\linespread{2}\selectfont}

%\makeatletter
%\renewcommand{\cftfigpresnum}{Abb. }
%\renewcommand{\cfttabpresnum}{Tab. }

%\setlength{\cftfignumwidth}{2cm}
%\setlength{\cfttabnumwidth}{2cm}

%\setlength{\cftfigindent}{0cm}
%\setlength{\cfttabindent}{0cm}
\makeatother

% CAPTION SETUP
\captionsetup{
    font=small,
    labelfont=bf,
    singlelinecheck=false,
    skip=10pt,
    belowskip=0pt
}

\renewcommand*{\labelalphaothers}{\textsuperscript{}}

% HEADERS & FOOTERS
\pagestyle{fancyplain}
\fancyhf{}
\renewcommand{\headrulewidth}{0pt}
\renewcommand{\footrulewidth}{0pt}
\setlength{\headheight}{15pt}

\fancyfoot[R]{\thepage} % Seitenzahlen unten rechts
\fancyfoot[L]{\leftmark} % linksbündig das Kapitel in der Fußzeile

% FOOTNOTE
%\renewcommand{\footnotelayout}{\hspace{0.5em}}

% Conditionals um AbbildungsVZ und TabellenVZ nur zu rendern, wenn sie nicht leer sind
%\newtotcounter{figCount}
%\newtotcounter{tabCount}
%\let\oldTabTOC=\table
%\let\oldFigTOC=\figure
%\renewcommand{\figure}{\stepcounter{figCount}\oldFigTOC}
%\renewcommand{\table}{\stepcounter{tabCount}\oldTabTOC}

%\newcommand{\conditionalLoF}{
%    \ifnum\totvalue{figCount}>0
%        \addcontentsline{toc}{section}{\listfigurename}
%        \listoffigures
%        \cleardoublepage
%    \fi
%}
%\newcommand{\conditionalLoT}{
%    \ifnum\totvalue{tabCount}>0
%        \addcontentsline{toc}{section}{\listtablename}
%        \listoftables
%        \cleardoublepage
%    \fi
%}

% Sections sollen mit Seitenumbruch beginnen
\let\stdsection\section
\renewcommand\section{\newpage\stdsection}

% Überschreiben von \mathrm{} -> einheitlichen Abstand einfügen
\let\oldMathrm\mathrm
\renewcommand{\mathrm}[1]{\,\oldMathrm{#1}}

% Template-Root-Verzeichnis ist das neue Arbeitsverzeichnis
% muss nach % DOCUMENT SETUP
\onehalfspacing
\widowpenalty10000
\clubpenalty10000

% INHALTSVERZEICHNIS SETUP
\cftsetindents	{section}{0em}{4em}
\cftsetindents	{subsection}{0em}{4em}
\cftsetindents	{subsubsection}{0em}{4em}
\renewcommand	{\contentsname}{Inhaltsverzeichnis}
\setcounter		{tocdepth}{3}
\setcounter		{secnumdepth}{5}

% ABBILDUNGEN UND TABELLEN SETUP

% Abbildungen usw. werden nach kapitel.lfd nummeriert
\counterwithin{figure}{section}
\counterwithin{table}{section}

%\renewcommand	{\listfigurename}{Abbildungsverzeichnis}
%\renewcommand	{\listtablename}{Tabellenverzeichnis}

%\addto{\captionsngerman}{
%    \renewcommand*{\figurename}{Abb.}
%    \renewcommand*{\tablename}{Tab.}
%}

%\addtocontents{lof}{\linespread{2}\selectfont}
%\addtocontents{lot}{\linespread{2}\selectfont}

%\makeatletter
%\renewcommand{\cftfigpresnum}{Abb. }
%\renewcommand{\cfttabpresnum}{Tab. }

%\setlength{\cftfignumwidth}{2cm}
%\setlength{\cfttabnumwidth}{2cm}

%\setlength{\cftfigindent}{0cm}
%\setlength{\cfttabindent}{0cm}
\makeatother

% CAPTION SETUP
\captionsetup{
    font=small,
    labelfont=bf,
    singlelinecheck=false,
    skip=10pt,
    belowskip=0pt
}

\renewcommand*{\labelalphaothers}{\textsuperscript{}}

% HEADERS & FOOTERS
\pagestyle{fancyplain}
\fancyhf{}
\renewcommand{\headrulewidth}{0pt}
\renewcommand{\footrulewidth}{0pt}
\setlength{\headheight}{15pt}

\fancyfoot[R]{\thepage} % Seitenzahlen unten rechts
\fancyfoot[L]{\leftmark} % linksbündig das Kapitel in der Fußzeile

% FOOTNOTE
%\renewcommand{\footnotelayout}{\hspace{0.5em}}

% Conditionals um AbbildungsVZ und TabellenVZ nur zu rendern, wenn sie nicht leer sind
%\newtotcounter{figCount}
%\newtotcounter{tabCount}
%\let\oldTabTOC=\table
%\let\oldFigTOC=\figure
%\renewcommand{\figure}{\stepcounter{figCount}\oldFigTOC}
%\renewcommand{\table}{\stepcounter{tabCount}\oldTabTOC}

%\newcommand{\conditionalLoF}{
%    \ifnum\totvalue{figCount}>0
%        \addcontentsline{toc}{section}{\listfigurename}
%        \listoffigures
%        \cleardoublepage
%    \fi
%}
%\newcommand{\conditionalLoT}{
%    \ifnum\totvalue{tabCount}>0
%        \addcontentsline{toc}{section}{\listtablename}
%        \listoftables
%        \cleardoublepage
%    \fi
%}

% Sections sollen mit Seitenumbruch beginnen
\let\stdsection\section
\renewcommand\section{\newpage\stdsection}

% Überschreiben von \mathrm{} -> einheitlichen Abstand einfügen
\let\oldMathrm\mathrm
\renewcommand{\mathrm}[1]{\,\oldMathrm{#1}}

% Template-Root-Verzeichnis ist das neue Arbeitsverzeichnis
% muss nach % DOCUMENT SETUP
\onehalfspacing
\widowpenalty10000
\clubpenalty10000

% INHALTSVERZEICHNIS SETUP
\cftsetindents	{section}{0em}{4em}
\cftsetindents	{subsection}{0em}{4em}
\cftsetindents	{subsubsection}{0em}{4em}
\renewcommand	{\contentsname}{Inhaltsverzeichnis}
\setcounter		{tocdepth}{3}
\setcounter		{secnumdepth}{5}

% ABBILDUNGEN UND TABELLEN SETUP

% Abbildungen usw. werden nach kapitel.lfd nummeriert
\counterwithin{figure}{section}
\counterwithin{table}{section}

%\renewcommand	{\listfigurename}{Abbildungsverzeichnis}
%\renewcommand	{\listtablename}{Tabellenverzeichnis}

%\addto{\captionsngerman}{
%    \renewcommand*{\figurename}{Abb.}
%    \renewcommand*{\tablename}{Tab.}
%}

%\addtocontents{lof}{\linespread{2}\selectfont}
%\addtocontents{lot}{\linespread{2}\selectfont}

%\makeatletter
%\renewcommand{\cftfigpresnum}{Abb. }
%\renewcommand{\cfttabpresnum}{Tab. }

%\setlength{\cftfignumwidth}{2cm}
%\setlength{\cfttabnumwidth}{2cm}

%\setlength{\cftfigindent}{0cm}
%\setlength{\cfttabindent}{0cm}
\makeatother

% CAPTION SETUP
\captionsetup{
    font=small,
    labelfont=bf,
    singlelinecheck=false,
    skip=10pt,
    belowskip=0pt
}

\renewcommand*{\labelalphaothers}{\textsuperscript{}}

% HEADERS & FOOTERS
\pagestyle{fancyplain}
\fancyhf{}
\renewcommand{\headrulewidth}{0pt}
\renewcommand{\footrulewidth}{0pt}
\setlength{\headheight}{15pt}

\fancyfoot[R]{\thepage} % Seitenzahlen unten rechts
\fancyfoot[L]{\leftmark} % linksbündig das Kapitel in der Fußzeile

% FOOTNOTE
%\renewcommand{\footnotelayout}{\hspace{0.5em}}

% Conditionals um AbbildungsVZ und TabellenVZ nur zu rendern, wenn sie nicht leer sind
%\newtotcounter{figCount}
%\newtotcounter{tabCount}
%\let\oldTabTOC=\table
%\let\oldFigTOC=\figure
%\renewcommand{\figure}{\stepcounter{figCount}\oldFigTOC}
%\renewcommand{\table}{\stepcounter{tabCount}\oldTabTOC}

%\newcommand{\conditionalLoF}{
%    \ifnum\totvalue{figCount}>0
%        \addcontentsline{toc}{section}{\listfigurename}
%        \listoffigures
%        \cleardoublepage
%    \fi
%}
%\newcommand{\conditionalLoT}{
%    \ifnum\totvalue{tabCount}>0
%        \addcontentsline{toc}{section}{\listtablename}
%        \listoftables
%        \cleardoublepage
%    \fi
%}

% Sections sollen mit Seitenumbruch beginnen
\let\stdsection\section
\renewcommand\section{\newpage\stdsection}

% Überschreiben von \mathrm{} -> einheitlichen Abstand einfügen
\let\oldMathrm\mathrm
\renewcommand{\mathrm}[1]{\,\oldMathrm{#1}}

% Template-Root-Verzeichnis ist das neue Arbeitsverzeichnis
% muss nach % DOCUMENT SETUP
\onehalfspacing
\widowpenalty10000
\clubpenalty10000

% INHALTSVERZEICHNIS SETUP
\cftsetindents	{section}{0em}{4em}
\cftsetindents	{subsection}{0em}{4em}
\cftsetindents	{subsubsection}{0em}{4em}
\renewcommand	{\contentsname}{Inhaltsverzeichnis}
\setcounter		{tocdepth}{3}
\setcounter		{secnumdepth}{5}

% ABBILDUNGEN UND TABELLEN SETUP

% Abbildungen usw. werden nach kapitel.lfd nummeriert
\counterwithin{figure}{section}
\counterwithin{table}{section}

%\renewcommand	{\listfigurename}{Abbildungsverzeichnis}
%\renewcommand	{\listtablename}{Tabellenverzeichnis}

%\addto{\captionsngerman}{
%    \renewcommand*{\figurename}{Abb.}
%    \renewcommand*{\tablename}{Tab.}
%}

%\addtocontents{lof}{\linespread{2}\selectfont}
%\addtocontents{lot}{\linespread{2}\selectfont}

%\makeatletter
%\renewcommand{\cftfigpresnum}{Abb. }
%\renewcommand{\cfttabpresnum}{Tab. }

%\setlength{\cftfignumwidth}{2cm}
%\setlength{\cfttabnumwidth}{2cm}

%\setlength{\cftfigindent}{0cm}
%\setlength{\cfttabindent}{0cm}
\makeatother

% CAPTION SETUP
\captionsetup{
    font=small,
    labelfont=bf,
    singlelinecheck=false,
    skip=10pt,
    belowskip=0pt
}

\renewcommand*{\labelalphaothers}{\textsuperscript{}}

% HEADERS & FOOTERS
\pagestyle{fancyplain}
\fancyhf{}
\renewcommand{\headrulewidth}{0pt}
\renewcommand{\footrulewidth}{0pt}
\setlength{\headheight}{15pt}

\fancyfoot[R]{\thepage} % Seitenzahlen unten rechts
\fancyfoot[L]{\leftmark} % linksbündig das Kapitel in der Fußzeile

% FOOTNOTE
%\renewcommand{\footnotelayout}{\hspace{0.5em}}

% Conditionals um AbbildungsVZ und TabellenVZ nur zu rendern, wenn sie nicht leer sind
%\newtotcounter{figCount}
%\newtotcounter{tabCount}
%\let\oldTabTOC=\table
%\let\oldFigTOC=\figure
%\renewcommand{\figure}{\stepcounter{figCount}\oldFigTOC}
%\renewcommand{\table}{\stepcounter{tabCount}\oldTabTOC}

%\newcommand{\conditionalLoF}{
%    \ifnum\totvalue{figCount}>0
%        \addcontentsline{toc}{section}{\listfigurename}
%        \listoffigures
%        \cleardoublepage
%    \fi
%}
%\newcommand{\conditionalLoT}{
%    \ifnum\totvalue{tabCount}>0
%        \addcontentsline{toc}{section}{\listtablename}
%        \listoftables
%        \cleardoublepage
%    \fi
%}

% Sections sollen mit Seitenumbruch beginnen
\let\stdsection\section
\renewcommand\section{\newpage\stdsection}

% Überschreiben von \mathrm{} -> einheitlichen Abstand einfügen
\let\oldMathrm\mathrm
\renewcommand{\mathrm}[1]{\,\oldMathrm{#1}}

% Template-Root-Verzeichnis ist das neue Arbeitsverzeichnis
% muss nach \input{setup.tex} aufgerufen werden, damit alle Template-Root-Dateien integriert werden
\makeatletter
\def\input@path{{../}{path1/}}
\makeatother

\newcounter{totalbibentries}
\newcommand*{\listcounted}{}

\makeatletter
\AtDataInput{
  \xifinlist{\abx@field@entrykey}\listcounted
    {}
    {\stepcounter{totalbibentries}
     \listxadd\listcounted{\abx@field@entrykey}}
}
\makeatother

% Abstände und Einrückungen abhängig von config.tex ein-/ausschalten
\if\CEINR 0
    \setlength{\parskip}{6pt}
    \setlength{\parindent}{0cm}
\fi
 aufgerufen werden, damit alle Template-Root-Dateien integriert werden
\makeatletter
\def\input@path{{../}{path1/}}
\makeatother

\newcounter{totalbibentries}
\newcommand*{\listcounted}{}

\makeatletter
\AtDataInput{
  \xifinlist{\abx@field@entrykey}\listcounted
    {}
    {\stepcounter{totalbibentries}
     \listxadd\listcounted{\abx@field@entrykey}}
}
\makeatother

% Abstände und Einrückungen abhängig von config.tex ein-/ausschalten
\if\CEINR 0
    \setlength{\parskip}{6pt}
    \setlength{\parindent}{0cm}
\fi
 aufgerufen werden, damit alle Template-Root-Dateien integriert werden
\makeatletter
\def\input@path{{../}{path1/}}
\makeatother

\newcounter{totalbibentries}
\newcommand*{\listcounted}{}

\makeatletter
\AtDataInput{
  \xifinlist{\abx@field@entrykey}\listcounted
    {}
    {\stepcounter{totalbibentries}
     \listxadd\listcounted{\abx@field@entrykey}}
}
\makeatother

% Abstände und Einrückungen abhängig von config.tex ein-/ausschalten
\if\CEINR 0
    \setlength{\parskip}{6pt}
    \setlength{\parindent}{0cm}
\fi
 aufgerufen werden, damit alle Template-Root-Dateien integriert werden
\makeatletter
\def\input@path{{../}{path1/}}
\makeatother

\newcounter{totalbibentries}
\newcommand*{\listcounted}{}

\makeatletter
\AtDataInput{
  \xifinlist{\abx@field@entrykey}\listcounted
    {}
    {\stepcounter{totalbibentries}
     \listxadd\listcounted{\abx@field@entrykey}}
}
\makeatother

% Abstände und Einrückungen abhängig von config.tex ein-/ausschalten
\if\CEINR 0
    \setlength{\parskip}{6pt}
    \setlength{\parindent}{0cm}
\fi
